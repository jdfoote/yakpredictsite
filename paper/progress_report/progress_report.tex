\documentclass{sigchi}

% Remove or comment out these two lines for final version
%\toappearbox{\Large Submitted to CHI'13. \\Do not cite, do not circulate.}
%\pagenumbering{arabic}% Arabic page numbers for submission. 

% Use \toappear{...} to override the default ACM copyright statement (e.g. for
% preprints).

% Load basic packages
\usepackage{balance}  % to better equalize the last page
\usepackage{graphics} % for EPS, load graphicx instead 
\usepackage{times}    % comment if you want LaTeX's default font 
\usepackage{url}      % llt: nicely formatted URLs

% llt: Define a global style for URLs, rather that the default one
\makeatletter \def\url@leostyle{%
\@ifundefined{selectfont}{\def\UrlFont{\sf}}{\def\UrlFont{\small\bf\ttfamily}}}
\makeatother \urlstyle{leo}


% To make various LaTeX processors do the right thing with page size.
\def\pprw{8.5in} \def\pprh{11in} \special{papersize=\pprw,\pprh}
\setlength{\paperwidth}{\pprw} \setlength{\paperheight}{\pprh}
\setlength{\pdfpagewidth}{\pprw} \setlength{\pdfpageheight}{\pprh}

% Make sure hyperref comes last of your loaded packages, to give it a fighting
% chance of not being over-written, since its job is to redefine many LaTeX
% commands.
\usepackage[pdftex]{hyperref} \hypersetup{ pdftitle={SIGCHI Conference
Proceedings Format}, pdfauthor={LaTeX}, pdfkeywords={SIGCHI, proceedings,
archival format}, bookmarksnumbered, pdfstartview={FitH}, colorlinks,
citecolor=black, filecolor=black, linkcolor=black, urlcolor=black,
breaklinks=true, }

% create a shortcut to typeset table headings
\newcommand\tabhead[1]{\small\textbf{#1}}


% End of preamble. Here it comes the document.
\begin{document}

\title{Yik Yak Progress Report - EECS 349}

% Note that submissions are blind, so author information should be omitted
\numberofauthors{2} \author{ \alignauthor Matthew Heston\\ Northwestern
University \alignauthor Jeremy Foote\\ Northwestern University }

% Teaser figure can go here
%\teaser{ \centering \includegraphics{Figure1} \caption{Teaser Image}
%\label{fig:teaser} }
\toappear{} \maketitle

%	Start by succinctly defining your task in terms of its inputs and outputs. For
%	example, ``our task is to predict the outcomes of baseball games based on
%	previous game data.'' Then, briefly (2-3 sentences) say why your task is
%	important.  Describe the data set you have utilized to date. What types of
%	attributes are there, how many attributes, how many examples, and how have you
%	partitioned the data for the purpose of
%	development/training/validation/testing.  Present your preliminary results on
%	the task. Which learning techniques have you tried, and how have they
%	performed? Note: Mention which existing machine learning software packages, if
%	any, you are utilizing. You can use any existing packages you like for the
%	project. Implementing learning algorithms can be, but does not need to be,
%	part of the work you do for the project.  Briefly (1-2 paragraphs) describe
%	your plans for the remainder of the quarter, and list any questions or
%	concerns you have. 


\section{Task}

Our task is to predict the score that a given ``yak'' would receive on the
anonymous mobile social app Yik Yak. Yik Yak is an anonymous, location-aware
mobile application in which users can create short posts, and can view and
upvote or downvote posts created near their location. While anonymous,
location-based messages have been a part of the world for a long time (e.g.,
writing on the bathroom stall), the particular combination of anonymity and mass
social feedback is enabled by GPS-enabled smartphones. Our goal is to understand
how people are using this new ``place'', and whether the norms and uses differ
by location.

\section{Data}

In order to make data exploration simpler, we have decided to look at the yaks
from two different campuses - Northwestern University and Florida State
University. Northwestern was chosen because we are likely to understand local
references and more easily interpret results of why some words or phrases are
associated with low or high scores. FSU was chosen to offer a contrast in that
it is a large state school. It is also the campus we have the most number of
yaks for, making it useful as we have a large amount of training data. We have
approximately 40,000 NU yaks and approximately 160,000 FSU yaks. For current
experiments, we have discretized score using fairly naive approaches, i.e.,
binning yaks based on somewhat arbitrary, though intuitive, thresholds (for
example, binning all yaks that receive a score of less than 0.)

\subsection{Features}

Our primary features are derived from the message text itself. We have primarily
used bag of words representations in our experiments up to thise point, though
we are considering deriving other features from the text (e.g., length of the
post.) We also have the timestamp of when the date was posted, which we have not
utilized yet, but have considered using to derive features as well (e.g., time
of day posted.)

\subsection{Partitioning}

We have been using 5-fold cross validation in our experiments up to this point,
but we also have held out data to test later. Also, our data includes time and
date information and spans about 5 months. This is useful as we may want to
train on earlier months and test on later months, as this more closely resembles
how an ML system would work ``in the wild.''

\section{Preliminary Results}

We have used word vector representations, using both unigram and bigram models,
and trained both naive bayes and logistic regression models as implemented in
the Python module sklearn. We have been using both accuracy and F-score as
metrics given the skewed nature of our class distributions. Our best average
F-score is 67.0\% from 5-fold cross validation using a bigram bag of words
representation and training a logistic regression model.

So far, we have tried a bag of words approach to create an estimated score for
yaks. We use Python tools for our analysis. We use pandas to store the data, and
sklearn tools to create models.

\section{Future Work}

For the remainder of the quarter we plan to:
\begin{itemize}
\item Incorporate other types of features (time of day, etc.) and see how they
affect performance of our classifiers.
\item Extend current word vector approaches (e.g., try tf-idf, counts of words
vs. binary whether or not a word exists, etc.) to see if this affects classifier
performance.
\item Compare features between NU and FSU models to see if they give us
insight into differences between campuses.
\end{itemize}

% \bibliographystyle{acm-sigchi} \bibliography{../bibfile}

\end{document}
